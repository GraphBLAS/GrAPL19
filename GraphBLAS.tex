\section{The GraphBLAS}
\label{sec:math}

%%  <<< Self Plagiarism Warning >>>>
%% The text from here to the end of the operations table was lifted from our
%% GABB paper where we introduced the C API
Consider a graph represented as an $n$-by-$n$ adjacency matrix $\matrix{A}$,
where $A_{ij}$ is the weight of the edge from vertex $i$ to vertex $j$,
and a second $k$-by-$n$ matrix $\matrix{B}$ representing a subset (of size $k$) of the vertices
in the graph, such that $B_{hi}$ is $1$ if the $h$th element of the subset is vertex $i$
(and all other elements of $\matrix{B}$ are 0).  The traditional matrix
product $\matrix{B} \times \matrix{A}$ over real arithmetic of these two matrices returns 
the cost based on the edge weights of reaching the set of vertices
adjacent to the vertices in $\matrix{B}$.  This fundamental operation can be
used to construct a wide range of graph algorithms.

We extend the range of graph operations by keeping the basic
pattern of a matrix-matrix multiplication, but varying
the operators and the interpretation of the values in the matrices (the \emph{domain}).
By carefully choosing operators and the domain, we control the
relation between matrix operations familiar in linear algebra and graph operations, thereby enabling
composable graph algorithms.

In addition to matrix multiplication, the GraphBLAS math specification defines
a range of additional operations over matrices and vectors.  These are summarized in Table~\ref{Tab:GraphBLASOps}.

\begin{table}[h]
\hrule
\begin{center}
\caption{A mathematical overview of the fundamental GraphBLAS operations supported
in the specification. $\matrix{A}$, $\matrix{B}$, and $\matrix{C}$ are GraphBLAS matrices; 
$\vector{u}$, $\vector{v}$, and $\vector{w}$ are GraphBLAS vectors; $i$ and $j$ are single indices;
$\mathbf{i}$ and $\mathbf{j}$ are arrays of indices;
$\oplus$ and $\otimes$ are arbitrary element-wise operators; the element-wise $\odot$
operator is used for the optional accumulation with the output GraphBLAS object where 
$x~\odot\hspace{-0.11cm}= y$ implies $x = x \odot y$; and $F_u()$ is a unary function.
Although not shown here, the input 
matrices $\matrix{A}$ and $\matrix{B}$ may be selected for transposition prior to 
the operation, and masks can be used to control which values are written to the output GraphBLAS object.
\label{Tab:GraphBLASOps}}
\newcommand{\odotequals}{~\odot\hspace{-0.09cm}=\hspace{-0.2cm}}
\begin{tabular}{l|rl}
{\sf Operation name} & \multicolumn{2}{c}{Mathematical description}  \\
\hline
{\sf mxm}          & $\matrix{C}$    $\odotequals$ & $\matrix{A} \oplus.\otimes \matrix{B}$ \\
{\sf mxv}          & $\vector{w}$    $\odotequals$ & $\matrix{A} \oplus.\otimes \vector{v}$ \\
{\sf vxm}          & $\vector{w}^T$  $\odotequals$ & $\vector{v}^T \oplus.\otimes \matrix{A}$  \\
{\sf eWiseMult}    & $\matrix{C}$    $\odotequals$ & $\matrix{A} \otimes \matrix{B}$ \\
                   & $\vector{w}$    $\odotequals$ & $\vector{u} \otimes \vector{v}$ \\
{\sf eWiseAdd}     & $\matrix{C}$    $\odotequals$ & $\matrix{A} \oplus \matrix{B}$ \\
                   & $\vector{w}$    $\odotequals$ & $\vector{u} \oplus \vector{v}$ \\
{\sf reduce} (row) & $\vector{w}$    $\odotequals$ & $\bigoplus_j\matrix{A}(:,j)$  \\
{\sf apply}        & $\matrix{C}$    $\odotequals$ & $F_u(\matrix{A})$ \\
                   & $\vector{w}$    $\odotequals$ & $F_u(\vector{u})$ \\
{\sf transpose}    & $\matrix{C}$    $\odotequals$ & $\matrix{A}^T$ \\
{\sf extract}      & $\matrix{C}$    $\odotequals$ & $\matrix{A}(\vector{i},\vector{j})$ \\
                   & $\vector{w}$    $\odotequals$ & $\vector{u}(\vector{i})$ \\
{\sf assign}       & $\matrix{C}(\vector{i},\vector{j})$  $\odotequals$ &  $\matrix{A}$ \\
                   & $\vector{w}(\vector{i})$  $\odotequals$ & $\vector{u}$ 
\end{tabular}

\end{center}
\hrule
\end{table}

%% << end of self-plagrism >>>

In mapping the GraphBLAS as a set of mathematical operators onto the C programming language
we made a number of fundamental choices~\cite{cspec}.  First, the core data structures required
to represent the objects defined by the GraphBLAS are opaque.   The GraphBLAS API defines a 
contract with the programmer for how these objects will be used, but the implementations and underlying
data structures are left to the implementation.  This opaqueness is critical if the API is to serve diverse hardware
ranging from CPUs to GPUs to specialized graph hardware. Second, we defined a non-blocking  
execution model that allows lazy evaluation.  Ultimately, to optimize sparse linear algebra software we need
to aggressively fuse operations and even restructure algorithms.  This requirement meant that we had to 
carefully define when results from a sequence of GraphBLAS operations must be materialized.

Since the release of the GraphBLAS specification, several implementations of the GraphBLAS have been
developed. These are briefly described below.

% SuiteSparse GraphBLAS:

\subsection{SuiteSparse:GraphBLAS}

SuiteSparse:GraphBLAS is a full implementation of the GraphBLAS standard, which
defines a set of sparse matrix operations on an extended algebra of semirings
using an almost unlimited variety of operators and types.  When applied to
sparse adjacency matrices, these algebraic operations are equivalent to
computations on graphs.  GraphBLAS provides a powerful and expressive framework
for creating graph algorithms based on the elegant mathematics of sparse matrix
operations on a semiring.

The design of a GraphBLAS library is flexible, because its data structures are
opaque to the user.  SuiteSparse:GraphBLAS uses a compressed-sparse vector
storage mechanism, in four different forms.  A matrix can be stored in
row-major order (CSR), or column-major order (CSC).  Each sparse vector
consists of a sorted list of indices, and the corresponding numerical values.
The sparse vectors are packed together into two arrays, and another ``pointer
array'' (of size equal to the dimension of the matrix, say $n$) keeps track of
where each row (or column) starts.  The space taken is $O(n+e)$ for a CSR
matrix with $n$ rows or a CSC matrix with $n$ columns, and with $e$ entries.
Both of these two forms can be modified in a {\em hypersparse} form
\cite{BulucGilbert08}, where the pointer array itself becomes sparse, and
non-empty vectors take no space at all.  The space is reduced to $O(e)$,
however, so that matrices with enormous dimensions can be created, as long as
$e << n$.  Hypersparsity is exploited automatically.  All methods can operate
on all four matrix formats in any combination.

The ability to incrementally modify a graph is critical in many applications.
GraphBLAS includes two operations that can make small incremental changes to a
graph/matrix:  namely \verb'GrB_setElement' and \verb'GrB_assign', and It would
be exceedingly slow to insert or delete a single new entry in a CSR or CSC
format, taking $O(n+e)$ time {\bf per entry} inserted.  Instead, the
non-blocking idea in GraphBLAS is exploited.  Fast deletion of entries is
handled by creating {\em zombies}, which are entries tagged for later deletion.
Fast insertion is handled with {\em pending tuples}, which is a separate
unordered list of $(i,j,a_{ij})$ for each new entry.  When a matrix operation
occurs (such as matrix multiply), all zombies are killed and all pending tuples
are assembled, in a single $O(n+ e + p \log p)$ step (for $p$ pending tuples),
or $O(e +p \log p)$ in the hypersparse case.  As a result, it is just as fast
to use a sequence of $e$ \verb'GrB_setElement' operations to build a matrix, as
it is to create an array of $e$ tuples and use \verb'GrB_build'.  Internally,
SuiteSparse:GraphBLAS is building the list itself, for the user, and then does
a \verb'GrB_build' when the matrix needs to be completed.

To enable high-performance matrix-matrix multiply, a code generation mechanism
is used to build functions for each semiring that can be created with built-in
operators.  The functions can rely on Gustavson's method \cite{Gustavson78}, a
dot product method, and heap-based method, all with masked variants.  A current
prototype of the package adds an early exit mechanism for the MIN, MAX, OR, and
AND monoids, where a dot product can terminate as soon as a terminal value is
found in the result (\verb'true' for OR, for example).  This will enable a fast
direct-optimizing BFS to be written, where the ``pull'' is a dot product, and
the ``push'' a saxpy-based operation (Gustavson's or the heap method).

Since it is meant as the GraphBLAS reference implementation, testing is a vital
component to the package.  In SuiteSparse:GraphBLAS, each GraphBLAS operation
was written twice: once in high-performance algorithms in C, and again in a
very simple and short MATLAB script, using dense matrices with of the required
type.  For example, \verb'GrB_assign' requires about 3,908 lines of C (not
counting comments), but only 161 lines in MATLAB.  The pattern in the MATLAB
version is held as a seperate boolean matrix.  The MATLAB functions are not
intended to be fast.  Instead, they exactly mimic the GraphBLAS API
Specification, line by line, so they can be visually inspected for conformance
to the spec.  For example, matrix multiply is written with a triply-nested
\verb'for' loop.  Then, to test the package, each computation is done in
SuiteSparse:GraphBLAS (via a MATLAB interface) and in the MATLAB mimic.  The
tests pass only if the results are identical in both value and pattern (even
with identical floating-point roundoff error, in most cases).

SuiteSparse:GraphBLAS appears in Debian and Ubuntu linux distros,
and has been released as part of the RedisGraph database
module of the Redis database systems, by RedisLabs, Inc.
\url{https://redislabs.com/redis-enterprise/redis-modules/redis-enterprise-modules/redisgraph/}



% IBM GraphBLAS:
\subsection{IBM GraphBLAS}

%% ToDo: Manoj and Jose

TODO: Manoj and Jose: Describe the IBM GraphBLAS implementation. Citations:\cite{gpi2016,GPI}



% GBTL: GraphBLAS Template Library
\subsection{GBTL: GraphBLAS Template Library}
 
The first version of the GraphBLAS Template Library (GBTL) was written in C++ 
when the GraphBLAS C API project was just beginning.  It was used, in part, 
to study early ideas under discussion in the specification process and was released as a proof of concept 
prior to the finalization of the GraphBLAS API Specification \cite{gbtl-cuda16, McMillan2016}. With the 
release of the GraphBLAS C API Specification~\cite{cspec} in 2017, GBTL was updated to conform to 
the mathematical behavior defined by the specification and released as version 2.0~\cite{gbtl-github}.  
Unlike the C API Specification, GBTL is written in C++ and makes judicious use of templates to 
make the generic aspects of the GraphBLAS specification easier to implement and 
more natural for the C++ programming language. When the GraphBLAS language committee
starts its work on the C++ language binding to the GraphBLAS, GBTL will be submitted as a 
proposed starting point for the discussion.

 
Central to GBTL's design is the concept of a separation of concerns between 
implementation of algorithms written in terms of the GraphBLAS primitives  
and the implementation of those primitives on a targeted hardware platform.   
This separation of concerns is defined by the GraphBLAS API Specification as
illustrated in Figure~\ref{fig:overview}.  
Above this API, GBTL has developed and includes a collection of graph algorithms 
written against its C++ API and has already been shown to be easily translated to 
the C API (compare Figures~\ref{code:pyGB}(c) and~\ref{code:pyGB}(d)).  Below the separation/API, different implementations of the GraphBLAS 
library can be supported for different hardware architectures (referred to as 
``backends'').  In this way we verify that algorithms written once against the API 
can run on different targeted hardware.  One backend that is provided with Version 
2.0 of GBTL implements a mathematically correct version of the C API 
specification and serves as a reference implementation for verifying correctness.  It runs 
in a single thread on a CPU (an earlier version of GBTL also had a GPU 
implementation).  Other backend implementations are currently under development 
to optimize performance, use multiple threads, and to target specialized 
computer architectures.

\subsection{PyGB: python DSL for GraphBLAS}

Another development effort closely related to GBTL is a DSL (domain-specific language) in Python called PyGB~\cite{Chamberlin2016}.  
The goal for PyGB is to closely resemble the GraphBLAS mathematical notation found in the GraphBLAS math spec~\cite{mathgraphblas16}.  
PyGB is a framework is designed and implemented to dispatch dynamically generated and compiled templated 
classes that make calls into native GBTL code.  It demonstrates how Python's syntax and dynamic execution provides 
a high-level abstraction with minimal performance penalty.  While we leave a detailed discussion of pyGB to elsewhere~\cite{Chamberlin2016}
we provide a level-BFS function using pyGB in Figure~\ref{code:pyGB}.

\begin{figure}

\begin{subfigure}{\columnwidth}
  \begin{cplus}
Input: graph, frontier, levels
depth |$\leftarrow$| 0
while nvals(frontier) > 0:
  depth |$\leftarrow$| depth + 1
  levels[frontier] |$\leftarrow$| depth
  frontier<|$\neg$|levels,replace> |$\leftarrow$| graph|$^T$| |$\oplus . \otimes$| frontier
    where |$\oplus . \otimes = \mathbf{\bigoplus} . \mathbf{\bigotimes}($|LogicalSemiring|$)$|
  \end{cplus}
%, |$\oplus  = \mathbf{\bigoplus}($|LogicalSemiring|$)$|
  \caption{Pseudocode}
  \label{subfig:pseudo}
\end{subfigure}\\[1ex]

\vspace{-0.6ex}

\begin{subfigure}{\columnwidth}
  \begin{python}
def bfs(graph, frontier, levels):
  depth = 0
  while frontier.nvals > 0:
    depth += 1
    levels[frontier][:] = depth
    with gb.LogicalSemiring, gb.Replace:
      frontier[~levels] = graph.T @ frontier
  \end{python}
 \caption{PyGB}
\end{subfigure}\\[1ex]

\begin{subfigure}{\columnwidth}
  \begin{cplus}
template<class Mat, class Frontier, class Levels>
void bfs(Mat &graph, Frontier frontier, Levels &levels)
{
  GrB::IndexType depth = 0;
  while (frontier.nvals() > 0) {
    ++depth;
    GrB::assign(levels, frontier, GrB::NoAccumulate(),
                depth, GrB::AllIndices(), false);
    GrB::mxv(frontier, GrB::complement(levels),
             GrB::NoAccumulate(),
             GrB::LogicalSemiring<GrB::IndexType>(),
             GrB::transpose(graph), frontier, true);
  }
}
\end{cplus}
\caption{GBTL \cpp}
\end{subfigure}

\begin{subfigure}{\columnwidth}
  \begin{cplus}
void bfs(GrB_Matrix  graph,
         GrB_Vector  frontier,
         GrB_Vector *levels)
{
  GrB_Index n, nvals;
  GrB_Matrix_nrows(&n, graph);
  GrB_Vector_nvals(&nvals, frontier);
  GrB_Semiring LogicalSemiring;
  GrB_Descriptor Desc_TranA_ScmpM_Replace;
  //...
  GrB_Index depth = 0;
  while (nvals > 0) {
    ++depth;
    GrB_assign(*levels, frontier, GrB_NULL,
               depth, GrB_ALL, n, GrB_NULL);
    GrB_mxv(frontier, *levels, GrB_NULL,
            LogicalSemiring, graph, frontier
            Desc_TranA_ScmpM_Replace);
    GrB_Vector_nvals(&nvals, frontier);
  }
}
\end{cplus}
\caption{GraphBLAS C API}
\end{subfigure}

\caption{Level-based BFS traversal in math pseudocode, PyGB, GBTL \cpp, and using the GraphBLAS C API.\label{code:pyGB}}
\end{figure}


Notice how the meaning of the code
is straightforward since the DSL closely tracks the notation from the GraphBLAS math spec.  We believe in the long run, the 
future of graph algorithms will depend heavily on such DSLs.

%\begin{CodeExample}
%{\textbf{Level-BFS function in pyGB}}
%{code:pyGB}
%\begin{lstlisting}
%def bfs(graph, frontier, levels):
%    level = 0
%    while frontier.nvals > 0:
%        level += 1
%        levels[front][:] = level
%        with gb.LogicalSemiring, gb.Replace:
%            frontier[~levels] = graph.T @ frontier 
%\end{lstlisting}
%\end{CodeExample}



% Gunrock GraphBLAS:
\subsection{Gunrock GraphBLAS}

%% ToDo: Carl

TODO: Carl: Describe the GunRock implementation



