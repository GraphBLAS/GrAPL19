\section{The GraphBLAS}
\label{sec:math}

%%  <<< Self Plagiarism Warning >>>>
%% The text from here to the end of the operations table was lifted from our
%% GABB paper where we introduced the C API
Consider a graph represented as an $n$-by-$n$ adjacency matrix $\matrix{A}$,
where $A_{ij}$ is the weight of the edge from vertex $i$ to vertex $j$,
and a second $k$-by-$n$ matrix $\matrix{B}$ representing a subset (of size k) of the vertices
in the graph, such that $B_{ji}$ is $1$ if the $j$th element of the subset is vertex $i$
(and all other elements of $\matrix{B}$ are 0).  The traditional matrix
product $\matrix{B} \times \matrix{A}$ over real arithmetic of these two matrices returns 
the cost based on the edge weights of reaching the set of vertices
adjacent to the vertices in $\matrix{B}$.  This fundamental operation can be
used to construct a wide range of graph algorithms.

We extend the range of graph operations by keeping the basic
pattern of a matrix-matrix multiplication, but varying
the operators and the interpretation of the values in the matrices (the \emph{domain}).
By carefully choosing operators and the domain, we control the
relation between matrix operations familiar in linear algebra and graph operations, thereby enabling
composable graph algorithms.

In addition to matrix multiplication, the GraphBLAS math specification defines
a range of additional operations over matrices and vectors.  These are summarized in Table~\ref{Tab:GraphBLASOps}.

\begin{table}[h]
\hrule
\begin{center}
\caption{A mathematical overview of the fundamental GraphBLAS operations supported
in this specification. $\matrix{A}$, $\matrix{B}$, and $\matrix{C}$ are GraphBLAS matrices; 
$\vector{u}$, $\vector{v}$, and $\vector{w}$ are GraphBLAS vectors; $i$ and $j$ are single indices;
$\mathbf{i}$ and $\mathbf{j}$ are arrays of indices;
$\oplus$ and $\otimes$ are arbitrary element-wise operators; the element-wise $\odot$
operator is used for the optional accumulation with the output GraphBLAS object where 
$x~\odot\hspace{-0.11cm}= y$ implies $x = x \odot y$; and $F_u()$ is a unary function.
Although not shown here, the input 
matrices $\matrix{A}$ and $\matrix{B}$ may be selected for transposition prior to 
the operation, and masks can be used to control which values are written to the output GraphBLAS object.}
\label{Tab:GraphBLASOps}
\newcommand{\odotequals}{~\odot\hspace{-0.09cm}=\hspace{-0.2cm}}
\begin{tabular}{l|rl}
{\sf Operation name} & \multicolumn{2}{c}{Mathematical description}  \\
\hline
{\sf mxm}          & $\matrix{C}$    $\odotequals$ & $\matrix{A} \oplus.\otimes \matrix{B}$ \\
{\sf mxv}          & $\vector{w}$    $\odotequals$ & $\matrix{A} \oplus.\otimes \vector{v}$ \\
{\sf vxm}          & $\vector{w}^T$  $\odotequals$ & $\vector{v}^T \oplus.\otimes \matrix{A}$  \\
{\sf eWiseMult}    & $\matrix{C}$    $\odotequals$ & $\matrix{A} \otimes \matrix{B}$ \\
                   & $\vector{w}$    $\odotequals$ & $\vector{u} \otimes \vector{v}$ \\
{\sf eWiseAdd}     & $\matrix{C}$    $\odotequals$ & $\matrix{A} \oplus \matrix{B}$ \\
                   & $\vector{w}$    $\odotequals$ & $\vector{u} \oplus \vector{v}$ \\
{\sf reduce} (row) & $\vector{w}$    $\odotequals$ & $\bigoplus_j\matrix{A}(:,j)$  \\
{\sf apply}        & $\matrix{C}$    $\odotequals$ & $F_u(\matrix{A})$ \\
                   & $\vector{w}$    $\odotequals$ & $F_u(\vector{u})$ \\
{\sf transpose}    & $\matrix{C}$    $\odotequals$ & $\matrix{A}^T$ \\
{\sf extract}      & $\matrix{C}$    $\odotequals$ & $\matrix{A}(\vector{i},\vector{j})$ \\
                   & $\vector{w}$    $\odotequals$ & $\vector{u}(\vector{i})$ \\
{\sf assign}       & $\matrix{C}(\vector{i},\vector{j})$  $\odotequals$ &  $\matrix{A}$ \\
                   & $\vector{w}(\vector{i})$  $\odotequals$ & $\vector{u}$ 
\end{tabular}

\end{center}
\hrule
\end{table}

%% << end of self-plagrism >>>

In mapping the GraphBLAS as a set of mathematical operators onto the C programming language
we made a number of fundamental choices~\cite{cspec}.  First, the core data structures required
to represent the objects defined by the GraphBLAS are opaque.   The GraphBLAS API defines a 
contract with the programmer for how these objects will be used, but the implementations and underlying
data structures are left to the implementation.  This opaqueness is critical if the API is to serve diverse hardware
ranging from CPUs to GPUs to specialized graph hardware. Second, we defined a non-blocking  
execution model that allows lazy evaluation.  Ultimately, to optimize sparse linear algebra software we need
to aggressively fuse operations and even restructure algorithms.  This requirement meant that we had to 
be very careful to define when results from a sequence of graphBLAS operations must be materialized.

Since the release of the GraphBLAS specification, several implementations of the graphBLAS have been
developed.

%% ToDo:  Tim Davis

Describe the SuiteSparse GraphBLAS

%% ToDo: Manoj and Jose

Describe the IBM GraphBLAS implementation

%% ToDo: Carl

Describe the GunRock implementation

%% ToDo: Scott, Tim M and Aydin

Describe C++ libraries consistent with the GraphBLAS such as GBTL and CompBLAS.

Describe Python binding.

