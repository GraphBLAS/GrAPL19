\section{The GraphBLAS}
\label{sec:math}

%%  <<< Self Plagiarism Warning >>>>
%% The text from here to the end of the operations table was lifted from our
%% GABB paper where we introduced the C API
Consider a graph represented as an $n$-by-$n$ adjacency matrix $\matrix{A}$,
where $A_{ij}$ is the weight of the edge from vertex $i$ to vertex $j$,
and a second $k$-by-$n$ matrix $\matrix{B}$ representing a subset (of size k) of the vertices
in the graph, such that $B_{ji}$ is $1$ if the $j$th element of the subset is vertex $i$
(and all other elements of $\matrix{B}$ are 0).  The traditional matrix
product $\matrix{B} \times \matrix{A}$ over real arithmetic of these two matrices returns 
the cost based on the edge weights of reaching the set of vertices
adjacent to the vertices in $\matrix{B}$.  This fundamental operation can be
used to construct a wide range of graph algorithms.

We extend the range of graph operations by keeping the basic
pattern of a matrix-matrix multiplication, but varying
the operators and the interpretation of the values in the matrices (the \emph{domain}).
By carefully choosing operators and the domain, we control the
relation between matrix operations familiar in linear algebra and graph operations, thereby enabling
composable graph algorithms.

In addition to matrix multiplication, the GraphBLAS math specification defines
a range of additional operations over matrices and vectors.  These are summarized in Table~\ref{Tab:GraphBLASOps}.

\begin{table}[h]
\hrule
\begin{center}
\caption{A mathematical overview of the fundamental GraphBLAS operations supported
in this specification. $\matrix{A}$, $\matrix{B}$, and $\matrix{C}$ are GraphBLAS matrices; 
$\vector{u}$, $\vector{v}$, and $\vector{w}$ are GraphBLAS vectors; $i$ and $j$ are single indices;
$\mathbf{i}$ and $\mathbf{j}$ are arrays of indices;
$\oplus$ and $\otimes$ are arbitrary element-wise operators; the element-wise $\odot$
operator is used for the optional accumulation with the output GraphBLAS object where 
$x~\odot\hspace{-0.11cm}= y$ implies $x = x \odot y$; and $F_u()$ is a unary function.
Although not shown here, the input 
matrices $\matrix{A}$ and $\matrix{B}$ may be selected for transposition prior to 
the operation, and masks can be used to control which values are written to the output GraphBLAS object.}
\label{Tab:GraphBLASOps}
\newcommand{\odotequals}{~\odot\hspace{-0.09cm}=\hspace{-0.2cm}}
\begin{tabular}{l|rl}
{\sf Operation name} & \multicolumn{2}{c}{Mathematical description}  \\
\hline
{\sf mxm}          & $\matrix{C}$    $\odotequals$ & $\matrix{A} \oplus.\otimes \matrix{B}$ \\
{\sf mxv}          & $\vector{w}$    $\odotequals$ & $\matrix{A} \oplus.\otimes \vector{v}$ \\
{\sf vxm}          & $\vector{w}^T$  $\odotequals$ & $\vector{v}^T \oplus.\otimes \matrix{A}$  \\
{\sf eWiseMult}    & $\matrix{C}$    $\odotequals$ & $\matrix{A} \otimes \matrix{B}$ \\
                   & $\vector{w}$    $\odotequals$ & $\vector{u} \otimes \vector{v}$ \\
{\sf eWiseAdd}     & $\matrix{C}$    $\odotequals$ & $\matrix{A} \oplus \matrix{B}$ \\
                   & $\vector{w}$    $\odotequals$ & $\vector{u} \oplus \vector{v}$ \\
{\sf reduce} (row) & $\vector{w}$    $\odotequals$ & $\bigoplus_j\matrix{A}(:,j)$  \\
{\sf apply}        & $\matrix{C}$    $\odotequals$ & $F_u(\matrix{A})$ \\
                   & $\vector{w}$    $\odotequals$ & $F_u(\vector{u})$ \\
{\sf transpose}    & $\matrix{C}$    $\odotequals$ & $\matrix{A}^T$ \\
{\sf extract}      & $\matrix{C}$    $\odotequals$ & $\matrix{A}(\vector{i},\vector{j})$ \\
                   & $\vector{w}$    $\odotequals$ & $\vector{u}(\vector{i})$ \\
{\sf assign}       & $\matrix{C}(\vector{i},\vector{j})$  $\odotequals$ &  $\matrix{A}$ \\
                   & $\vector{w}(\vector{i})$  $\odotequals$ & $\vector{u}$ 
\end{tabular}

\end{center}
\hrule
\end{table}

%% << end of self-plagrism >>>

In mapping the GraphBLAS as a set of mathematical operators onto the C programming language
we made a number of fundamental choices~\cite{cspec}.  First, the core data structures required
to represent the objects defined by the GraphBLAS are opaque.   The GraphBLAS API defines a 
contract with the programmer for how these objects will be used, but the implementations and underlying
data structures are left to the implementation.  This opaqueness is critical if the API is to serve diverse hardware
ranging from CPUs to GPUs to specialized graph hardware. Second, we defined a non-blocking  
execution model that allows lazy evaluation.  Ultimately, to optimize sparse linear algebra software we need
to aggressively fuse operations and even restructure algorithms.  This requirement meant that we had to 
be very careful to define when results from a sequence of graphBLAS operations must be materialized.

Since the release of the GraphBLAS specification, several implementations of the graphBLAS have been
developed.

%% ToDo:  Tim Davis

% Describe the SuiteSparse GraphBLAS

\subsection{SuiteSparse:GraphBLAS}

SuiteSparse:GraphBLAS is the first full implementation of the GraphBLAS
standard, first released in November of 2017.
It is available at \url{http://suitesparse.com} \cite{Davis19}.

The design of a GraphBLAS library is flexible, because its data structures are
opaque to the user.  SuiteSparse:GraphBLAS uses a compressed-sparse vector
data structure, in four different forms.  A matrix can be stored in
row-major order (CSR), or column-major order (CSC).  Each sparse vector
consists of a sorted list of indices, and the corresponding numerical values.
The sparse vectors are packed together into two arrays, and another ``pointer
array'' (of size equal to the dimension of the matrix, say $n$) keeps track of
where each row (or column) starts.  The memory taken is $O(n+e)$ for a CSR
matrix with $n$ rows or a CSC matrix with $n$ columns, and with $e$ entries.
Most graphs have $e=O(n)$ entries, but some graphs (and in particular,
subgraphs) can be {\em hypersparse} \cite{BulucGilbert08}, with $e << n$.  In
the hypersparse form, the pointer array itself becomes sparse, and empty
vectors take no space at all.  The space is reduced to $O(e)$, so that
matrices with enormous dimensions can be created, as long as $e << n$.
SuiteSparse:GraphBLAS exploits hypersparsity automatically, and all
methods can operate on all four matrix formats in any combination.

The ability to incrementally modify a graph is critical in many applications.
GraphBLAS includes two operations that can make small incremental changes to a
graph/matrix:  namely \verb'GrB_setElement' and \verb'GrB_assign'.  It would be
exceedingly slow to insert or delete a single entry in a CSR or CSC format,
taking $O(n+e)$ time {\bf per entry} inserted or deleted.  Instead, the
non-blocking aspect of GraphBLAS is exploited.  Fast deletion of entries is
handled by creating {\em zombies}, which are entries tagged for later deletion.
Fast insertion is handled with {\em pending tuples}, which is a separate
unordered list of $(i,j,a_{ij})$ for each new entry.  When a matrix operation
occurs (such as matrix multiply), all zombies are killed and all pending tuples
are assembled, in a single $O(n+ e + p \log p)$ step (for $p$ pending tuples),
or $O(e +p \log p)$ in the hypersparse case.  As a result, it is just as fast
to use a sequence of $e$ \verb'GrB_setElement' operations to build a matrix, as
it is to create an array of $e$ tuples and use \verb'GrB_build'.  Internally,
SuiteSparse:GraphBLAS is building the list itself, for the user, and then does
a \verb'GrB_build' when the matrix needs to be completed.

To enable high-performance matrix-matrix multiply, a code generation mechanism
is used to build functions for each semiring that can be created with built-in
operators.  The functions can rely on Gustavson's method \cite{Gustavson78}, a
dot product method, and a heap-based method \cite{sisc3dspgemm}, all with
masked variants.  A current prototype of the package adds an early exit
mechanism for the MIN, MAX, OR, and AND monoids, where a dot product can
terminate as soon as a terminal value is found in the result (\verb'true' for
OR, for example).  This will enable a fast direction-optimizing BFS
\cite{Beamer:2012:DOB} to be written in GraphBLAS.  The ``pull'' is a dot
product, and the ``push'' a saxpy-based operation (Gustavson's or the heap
method).

Since its creation was commissioned as the GraphBLAS reference implementation,
testing is a vital component to the package.  In SuiteSparse:GraphBLAS, each
GraphBLAS operation was written twice: once in high-performance algorithms in
C, and again in a very simple and short MATLAB script, using dense matrices
with the required type.  The pattern in the MATLAB version is held as a
separate Boolean matrix.  For example, \verb'GrB_assign' requires about 3,908
lines of C (not counting comments), but only 161 lines in MATLAB. Of those 161
lines, 33 are for error-checking that do not need to be considered when
determining conformance to the spec.  The MATLAB functions are not intended to
be fast.  Instead, they exactly mimic the GraphBLAS API Specification, line by
line, so they can be visually inspected for conformance to the spec.  For
example, matrix multiply is written with a brute-force triply-nested \verb'for'
loop.  Then, to test the package, each computation is done both in
SuiteSparse:GraphBLAS (via a MATLAB interface) and in the MATLAB mimic.  The
tests pass only if the results are identical in both value and pattern (even
with identical floating-point roundoff error, in most cases).

The current release is single-threaded, but an OpenMP implementation is in
progress.  SuiteSparse:GraphBLAS appears in Debian and Ubuntu Linux distros,
and has been released as part of the RedisGraph database module of the Redis
database systems, by RedisLabs, Inc. \cite{redisgraph}.



%% ToDo: Manoj and Jose

Describe the IBM GraphBLAS implementation

%% ToDo: Carl

Describe the GunRock implementation

%% ToDo: Scott, Tim M and Aydin

Describe C++ libraries consistent with the GraphBLAS such as GBTL and CompBLAS.

Describe Python binding.

