
\usepackage{microtype}
\renewcommand*\ttdefault{txtt}

\usepackage{relsize}
\usepackage{xspace}
\usepackage[normalem]{ulem}

% BNF Grammars
\usepackage[nounderscore]{syntax}

\usepackage{flushend}

%\usepackage{xcolor,colortbl}

%\usepackage{algorithmic}
%\renewcommand{\algorithmiccomment}[1]{\hfill \{#1\}}
\newcommand{\tid}{\mathit{tid}}
\newcommand{\pop}{\mathit{pop}}
\newcommand{\scount}{\mathit{send\_count}}

\definecolor{LightGray}{gray}{0.97}
\definecolor{MidGray}{gray}{0.9}
\definecolor{HeavyGray}{gray}{0.83}
\newcommand{\lcell}[3]{\multicolumn{1}{@{}>{\columncolor{#1}[0pt][\tabcolsep]}#2}{#3}}
\newcommand{\rcell}[3]{\multicolumn{1}{>{\columncolor{#1}[\tabcolsep][0pt]}#2@{}}{#3}}
\newcommand{\LGlcell}[1]{\lcell{LightGray}{l}{#1}}
\newcommand{\LGrcell}[1]{\rcell{LightGray}{l}{#1}}
\newcommand{\MGlcell}[1]{\lcell{MidGray}{l}{#1}}
\newcommand{\MGrcell}[1]{\rcell{MidGray}{l}{#1}}
\newcommand{\HGlcell}[1]{\lcell{HeavyGray}{l}{#1}}
\newcommand{\HGrcell}[1]{\rcell{HeavyGray}{l}{#1}}

%Use hyperref for linking and PDF bookmarks.
%The commented options allow adjusting colors of links if so desired.
\usepackage[
  bookmarksnumbered=true,
%  colorlinks=true,
  pdfstartview=FitV,
  bookmarksopen,
%  linkcolor=blue,
%  citecolor=blue,
%  urlcolor=blue,
  bookmarksopenlevel=2]
{hyperref}
\usepackage[all]{hypcap}
% \hypersetup{
%   bookmarksnumbered=true,
%   pdfstartview=FitV,
%   bookmarksopen,
%   linkcolor=blue,
%   citecolor=blue,
%   urlcolor=blue,
%   bookmarksopenlevel=2,
%   colorlinks=true, %set true if you want colored links
%   %% linktoc=all,     %set to all if you want both sections and subsections linked
%   %% linkcolor=blue,  %choose some color if you want links to stand out
%   }

\usepackage{wrapfig}

\usepackage{mdwlist}

% Define keywords and settings for C++
\lstdefinelanguage{C++1y}{
  alsolanguage=C++,
  escapechar=@,
  breakatwhitespace=true,
  morekeywords = {
    alignof, decltype, concept, axiom, requires, property
  }
}

% Define an environment for C++ programs
\lstnewenvironment{program}[1][\small]
{
  \lstset{
    style=C++,
    basicstyle=#1\sffamily,
    keywordstyle=#1\sffamily\bfseries,
    commentstyle=#1\sffamily\itshape,
  }
}
{ }

% Define an environment for C++ programs with non-bold keywords
\lstnewenvironment{programnb}[1][\small]
{
  \lstset{
    style=C++,
    basicstyle=#1\sffamily,
    keywordstyle=#1\sffamily,
    commentstyle=#1\sffamily,
  }
}
{ }


\usepackage{tikz}

% code listings
% \definecolor{lightblue}{rgb}{0.23,0.21,0.90}
\definecolor{lightblue}{rgb}{0,0,0}
\lstdefinelanguage{C++custom}
{%columns=fullflexible,
escapeinside={/*@}{@*/},breaklines=true,breakatwhitespace=true,%
basicstyle=\color{lightblue},keywordstyle=,%
lineskip=-.05\baselineskip,morekeywords={pattern,in,is,to,from,out_edges,adj,once,fixed_point,vertex_property,edge_property,vertex,edge,meta,auto,concept,requires,concept_map},%
alsolanguage=C++,
literate = %
  {\{}{\smaller{$\{$}}{1}%
  {\}}{\smaller{$\}$}}{1}%
  {<=}{$\leq$}{1}%
  {-}{--}{1}%
}

\lstdefinelanguage{Pythoncustom}
{%columns=fullflexible,
escapeinside={/*@}{@*/},breaklines=true,breakatwhitespace=true,%
basicstyle=\color{lightblue},keywordstyle=,%
lineskip=-.05\baselineskip,%morekeywords={}%pattern,in,is,to,from,out_edges,adj,once,fixed_point,vertex_property,edge_property,vertex,edge,meta,auto,concept,requires,concept_map},%
alsolanguage=Python,
literate = %
  {\{}{\smaller{$\{$}}{1}%
  {\}}{\smaller{$\}$}}{1}%
  {<=}{$\leq$}{1}%
  {-}{--}{1}%
}

\lstdefinestyle{numbers}
{xleftmargin=8pt,numbers=left, numberstyle=\tiny,%
stepnumber=1, numbersep=4pt}

\lstdefinestyle{frametb}
{frame=tb}

\lstdefinestyle{bold-keywords}{keywordstyle=\bfseries}

\lstnewenvironment{myverb}[1][\small]{\lstset{%language=C++custom,%
    %style=numbers,
    columns=fullflexible,
    basicstyle=\color{lightblue}#1\ttfamily,%
    %keywordstyle=#1\ttfamily,%
    %style=bold-keywords,
    breaklines=true,
    %style=frametb
  }}{}

\newif\ifminted
\mintedtrue
%\mintedfalse
\ifminted
\usepackage{minted}
\setminted{fontsize=\footnotesize,baselinestretch=.97,linenos,frame=lines,xleftmargin=6pt,numbersep=3pt,mathescape=true,escapeinside=||,bgcolor=bg}
\usemintedstyle{default}
\definecolor{bg}{rgb}{0.97,0.97,0.97}
\newminted{cpp}{}
\newenvironment{cplus}{\VerbatimEnvironment\begin{cppcode}}{\end{cppcode}}
\newmintinline[cplusinl]{cpp}{}
\newminted{python3}{}
\newenvironment{python}{\VerbatimEnvironment\begin{python3code}}{\end{python3code}}
\newmintinline[pytinl]{python3}{fontsize=\small,breaklines}
\else
\lstnewenvironment{cplus}[1][\footnotesize]{\lstset{language=C++custom,%
    style=numbers,basicstyle=\color{lightblue}#1\ttfamily,%
    keywordstyle=#1\ttfamily,%
    style=bold-keywords,style=frametb}}{}

\lstnewenvironment{cplusuf}[1][\small]{\lstset{language=C++custom,%
    style=numbers,basicstyle=\color{lightblue}#1\ttfamily,%
    keywordstyle=#1\ttfamily,%
    style=bold-keywords,frame=t%,style=frametb
  }}{}

\lstnewenvironment{mjava}[1][\small]{\lstset{language=Java,%
    escapeinside={/*@}{@*/},style=numbers,basicstyle=\color{lightblue}#1\ttfamily,%
    keywordstyle=#1\ttfamily,%
    style=bold-keywords,style=frametb}}{}

\lstnewenvironment{cplusnln}[1][\small]{\lstset{language=C++custom,%columns=flexible,%
    xleftmargin=8pt,xrightmargin=8pt,basicstyle=\color{lightblue}#1\ttfamily,%
    keywordstyle=#1\ttfamily,%
    style=bold-keywords,style=frametb}}{}

\makeatletter
\lstnewenvironment{code}[1][\footnotesize]{\lstset{language=C++custom,columns=fullflexible,%
    xleftmargin=8pt,xrightmargin=8pt,basicstyle=\color{lightblue}#1,%
    keywordstyle=#1,style=numbers,%
    style=bold-keywords,mathescape=true}}{\@endparenv}
\makeatother

\lstnewenvironment{python}[1][\footnotesize]{\lstset{language=Pythoncustom,%
    style=numbers,basicstyle=\color{lightblue}#1\ttfamily,%
    keywordstyle=#1\ttfamily,%
    style=bold-keywords,style=frametb}}{}

\lstnewenvironment{pythonuf}[1][\small]{\lstset{language=Pythoncustom,%
    style=numbers,basicstyle=\color{lightblue}#1\ttfamily,%
    keywordstyle=#1\ttfamily,%
    style=bold-keywords,frame=t%,style=frametb
  }}{}

\lstnewenvironment{pythonln}[1][\small]{\lstset{language=Pythoncustom,%columns=flexible,%
    xleftmargin=8pt,xrightmargin=8pt,basicstyle=\color{lightblue}#1\ttfamily,%
    keywordstyle=#1\ttfamily,%
    style=bold-keywords,style=frametb}}{}

\makeatletter
\lstnewenvironment{pcode}[1][\footnotesize]{\lstset{language=Pythoncustom,columns=fullflexible,%
    xleftmargin=8pt,xrightmargin=8pt,basicstyle=\color{lightblue}#1,%
    keywordstyle=#1,style=numbers,%
    style=bold-keywords,mathescape=true}}{\@endparenv}
\makeatother\fi



\providecommand{\codeinl}[2][\small] %[\normalsize]
{{\lstinline[language=C++custom,breaklines=false,columns=fullflexible,%
basicstyle=\color{lightblue}#1\sffamily%\itshape
,mathescape=true,keywordstyle=#1\sffamily]@#2@}}%

\providecommand{\cplusinl}[2][\small] %[\normalsize]
{{\lstinline[language=C++custom,breaklines=false,columns=fullflexible,%
basicstyle=\color{lightblue}#1\ttfamily%\itshape
,keywordstyle=#1\ttfamily]@#2@}}%

\providecommand{\pytinl}[2][\small] %[\normalsize]
{{\lstinline[language=Pythoncustom,breaklines=false,columns=fullflexible,%
basicstyle=\color{lightblue}#1\ttfamily%\itshape
,keywordstyle=#1\ttfamily]@#2@}}%

\lstdefinestyle{cppmarkers}{rangeprefix=/*\#\ ,%
includerangemarker=false,%
rangesuffix=\ \#*/}%

\providecommand{\cplusinput}[3][\footnotesize]{{\lstinputlisting[language=C++custom,style=cppmarkers,style=numbers,basicstyle=\color{lightblue}#1\ttfamily,keywordstyle=#1\ttfamily,style=bold-keywords,style=frametb,firstnumber=1,linerange={#3}]{#2}}}

\lstdefinelanguage{Haskell-custom}
{%columns=flexible,
escapeinside={--@}{@--},breaklines=true,breakatwhitespace=true%
language=Haskell,basicstyle=\color{lightblue}\ttfamily,keywordstyle=\ttfamily,%
morekeywords={class,instance,type,newtype,data,where,deriving,import},%
lineskip=-.1\baselineskip,morekeywords={concept,requires,concept_map},
literate={+}{{$+$}}1 {/}{{$/$}}1 {*}{{$*$}}1 {=}{{$=$}}1
               {>}{{$>$}}1 {<}{{$<$}}1 {\\}{{$\lambda$}}1
               {\\\\}{{\char`\\\char`\\}}1
               {->}{{$\rightarrow$}}2 {>=}{{$\geq$}}2 {<-}{{$\leftarrow$}}2
               {=>}{{$\Rightarrow$}}2
               {\ .}{{$\circ$}}2 {\ .\ }{{$\circ$}}2
               {>>}{{>>}}2 {>>=}{{>>=}}2
               {|}{{$\mid$}}1
             }

\lstnewenvironment{hask}[1][\small]{\lstset{language=Haskell-custom,%
    style=numbers,basicstyle=\color{lightblue}#1\ttfamily,keywordstyle=#1\ttfamily,%
    style=bold-keywords,style=frametb}}{}

\providecommand{\haskellinl}[2][\normalsize]{{\lstinline[language=Haskell-custom, columns=fullflexible%
basicstyle=\color{lightblue}#1\ttfamily,keywordstyle=#1\ttfamily]@#2@}}%

\providecommand{\haskinl}[2][\normalsize]{{\lstinline[language=Haskell-custom,columns=fullflexible,%
basicstyle=\color{lightblue}#1\ttfamily,mathescape=true,keywordstyle=#1\ttfamily]@#2@}}%

\lstdefinestyle{markers}{rangeprefix=\{-\:\ ,%
includerangemarker=false,%
rangesuffix=\ \:-\}}%

\providecommand{\haskin}[3][\footnotesize]{{\lstinputlisting[language=Haskell-custom,basicstyle=\color{lightblue}#1\ttfamily,keywordstyle=#1\ttfamily,%
style=bold-keywords,style=markers,style=numbers,style=frametb,firstnumber=1,linerange={#3}]{#2}}}

\providecommand{\haskellinputnonumber}[3][\small]{{\lstinputlisting[language=Haskell-custom,basicstyle=\color{lightblue}#1\ttfamily,keywordstyle=#1\ttfamily,%
style=bold-keywords,style=frametb,%xleftmargin=8pt,xrightmargin=8pt,
style=markers,firstnumber=1,linerange={#3}]{#2}}}

% Define the test style
% Uncomment the literate stuff for improved formatting. Note that changing
% the '_' token causes the lexer to tokenize keywords in underscore-separated
% identifiers., That's why we have some additional replacements.
\lstdefinestyle{C++}
{
  language=C++1y,
  columns=fullflexible,
  breaklines=true,
  % literate=
    % {-}{\ttfamily --}{1}
    % {_}{\underscore}{1}
    % {=}{\ttfamily =}{1}
    % {+}{\ttfamily +}{1}
    % {>}{\larger\ttfamily >}{1}
    % {<}{\larger\ttfamily <}{1}
    % {&}{\larger\ttfamily \&}{1}
    % {*}{\ttfamily *}{1}
    % {|}{\larger\ttfamily |}{1}
    % {for_}{for\underscore}{1}
    % {not_}{not\underscore}{1}
    % {_if}{\underscore if}{1}
    % {_not}{\underscore not}{1}
    % {_union}{\underscore union}{1}
    % {_mutable}{\underscore mutable}{1}
    % {_signed}{\underscore signed}{1}
    % {_unsigned}{\underscore unsigned}{1}
}

% Define a language and settings for CASL programs
\lstdefinelanguage{CASL}{
  escapechar=@,
  breakatwhitespace=true,
  morekeywords = {
    spec, sort, then, op, ops, var, vars, pred, end
  }
}

% Define an environment for CASL specifications.
\lstnewenvironment{casl}[1][\small]
{
  \lstset{
    language=CASL,
    basicstyle=#1\sffamily,
    keywordstyle=#1\sffamily\bfseries,
    commentstyle=#1\sffamily,
    columns=fullflexible,
    breaklines=true,
  }
}
{ }

% Taken from http://mintaka.sdsu.edu/GF/bibliog/latex/floats.html
% Alter some LaTeX defaults for better treatment of figures:
% See p.105 of "TeX Unbound" for suggested values.
% See pp. 199-200 of Lamport's "LaTeX" book for details.
%   General parameters, for ALL pages:
\renewcommand{\topfraction}{0.9}	% max fraction of floats at top
\renewcommand{\bottomfraction}{0.9}	% max fraction of floats at bottom
%   Parameters for TEXT pages (not float pages):
\setcounter{topnumber}{4}
\setcounter{bottomnumber}{4}
\setcounter{totalnumber}{4}     % 2 may work better
\setcounter{dbltopnumber}{4}    % for 2-column pages
\renewcommand{\dbltopfraction}{0.9}	% fit big float above 2-col. text
\renewcommand{\textfraction}{0.01}	% allow minimal text w. figs
%   Parameters for FLOAT pages (not text pages):
\renewcommand{\floatpagefraction}{0.7}	% require fuller float pages
% N.B.: floatpagefraction MUST be less than topfraction !!
\renewcommand{\dblfloatpagefraction}{0.7}	% require fuller float pages
% remember to use [htp] or [htpb] for placement


%\usepackage{eqparbox}

%%% Local Variables:
%%% mode: latex
%%% TeX-master: "paper"
%%% End:
