\documentclass[10pt, conference, compsocconf,letter]{IEEEtran}   	
%\usepackage{geometry}                		% See geometry.pdf to learn the layout options. There are lots.
%\geometry{letterpaper}                		% ... or a4paper or a5paper or ... 
%\geometry{landscape}                		% Activate for for rotated page geometry
%\usepackage[parfill]{parskip}    		% Activate to begin paragraphs with an empty line rather than an indent
\usepackage{graphicx}				% Use pdf, png, jpg, or eps§ with pdflatex; use eps in DVI mode
						% TeX will automatically convert eps --> pdf in pdflatex

\usepackage{color}
\newcommand{\fix}[1]{\textcolor{red}{#1}}
\newcommand{\highlight}[1]{\textcolor{red}{#1}}
\usepackage{amssymb, amsmath}
\usepackage{mathtools}
\usepackage{booktabs}
%\usepackage[noend]{algpseudocode}
%\usepackage[ruled,vlined]{algorithm2e}
\usepackage{verbatim}
\newcommand*\rot{\rotatebox{90}}
%\usepackage[compatible]{algpseudocode}
\usepackage{algorithm}
\usepackage[noend]{algpseudocode}
\usepackage{varwidth}
\usepackage{multirow}
\usepackage{rotating, xcolor}
\usepackage{subfig}
\usepackage{enumitem}
%\usepackage[caption=false]{subfig}
\usepackage[hyphens]{url}
\usepackage{float}
\newfloat{algorithm}{t}{lop}

\usepackage{tcolorbox}
\definecolor{mycolor}{rgb}{0.122, 0.435, 0.698}
%\begin{tcolorbox}[width=\linewidth, boxsep=0pt, left=-4pt, boxrule=0.5pt]
%\end{tcolorbox} 
%\newtcolorbox{mybox}{colback=red!5!white,colframe=mycolor}
\makeatletter
\newcommand{\mybox}[1]{%
  %\setbox0=\hbox{#1}%
  %\setlength{\@tempdima}{\dimexpr\wd0+13pt}%
  \begin{tcolorbox}[colframe=mycolor,boxrule=0.5pt,arc=4pt,
      left=-4pt,right=6pt,top=6pt,bottom=6pt,boxsep=0pt,width=\linewidth]
    #1
  \end{tcolorbox}
}
\makeatother

\algnewcommand{\LineComment}[1]{\State \(\triangleright\) #1}
%\newcommand{\procdecl}[1]   {\proc{#1}\vrule width0pt height0pt depth 7pt \relax}
\newcommand{\lilabel}[1]        {\label{li:#1}}

\newcommand{\erdosrenyi}{Erd\H os-R\'{e}nyi }
\newcommand{\qg}{\u{g}}
\newcommand{\qG}{\u{G}}
\newcommand{\qc}{\c{c}}
\newcommand{\qC}{\c{C}}
\newcommand{\qs}{\c{s}}
\newcommand{\qS}{\c{S}}
\newcommand{\qu}{\"{u}}
\newcommand{\qU}{\"{U}}
\newcommand{\qo}{\"{o}}
\newcommand{\qO}{\"{O}}
\newcommand{\qI}{\.{I}}
\newcommand{\wa}{\^{a}}
\newcommand{\wA}{\^{A}}
\newcommand{\xor}{\mbox{xor}}
\usepackage{mathtools}
\DeclarePairedDelimiter\ceil{\lceil}{\rceil}
\DeclarePairedDelimiter\floor{\lfloor}{\rfloor}

\renewcommand{\floatpagefraction}{.9}
\newcommand{\minusone}{\text{-}1}

%% ABAB: Shortcuts/macros below are not used but it can be referenced as a cheatsheet
\newcommand{\liref}[1]      {line~\ref{li:#1}}
\newcommand{\Liref}[1]      {Line~\ref{li:#1}}
\newcommand{\lirefs}[2]     {lines \ref{li:#1}--\ref{li:#2}}
\newcommand{\Lirefs}[2]     {Lines \ref{li:#1}--\ref{li:#2}}
\newcommand{\lireftwo}[2]   {lines \ref{li:#1} and~\ref{li:#2}}
\newcommand{\lirefthree}[3] {lines \ref{li:#1}, \ref{li:#2}, and~\ref{li:#3}}

\def\Cpp{C{}\texttt{++}}
\newcommand{\mA}{\mathbf{A}} 
\newcommand{\mL}{\mathbf{L}}
\newcommand{\mU}{\mathbf{U}}
\newcommand{\transpose}     {^{\mbox{\scriptsize \sf T}}}
\newcommand{\mB}{\mathbf{B}}
\newcommand{\mC}{\mathbf{C}}
\newcommand{\dimN}{n}
\newcommand{\dimM}{m}
\newcommand{\dimK}{k}
\newcommand{\dnzc}{\id{nzc}}
\newcommand{\dnzr}{\id{nzr}}
\newcommand{\dni}{\id{ni}}
\newcommand{\dnnz}{\id{nnz}}
\newcommand{\dsort}{\id{sort}}
\newcommand{\dscan}{\id{scan}}
\newcommand{\dsearch}{\id{search}}
\newcommand{\dmin}{\func{min}}
\newcommand{\dmax}{\func{max}}
\newcommand{\dth}{th}
\newcommand{\dlen}{\id{len}}
\newcommand{\matlab}{{\sc MATLAB}}

%
%TGM: included and new commands from our GraphBLAS C API document
%
\usepackage{listings}
\renewcommand{\vector}[1]{{\bf #1}}
\renewcommand{\matrix}[1]{{\bf #1}}
\renewcommand{\arg}[1]{{\sf #1}}
\newcommand{\zip}{{\mbox{zip}}}
\newcommand{\zap}{{\mbox{zap}}}
\newcommand{\ewiseadd}{{\mbox{\bf ewiseadd}}}
\newcommand{\ewisemult}{{\mbox{\bf ewisemult}}}
\newcommand{\mxm}{{\mbox{\bf mxm}}}
\newcommand{\vxm}{{\mbox{\bf vxm}}}
\newcommand{\mxv}{{\mbox{\bf mxv}}}
\newcommand{\gpit}[1]{{\sf #1}}
\newcommand{\ie}{\emph{i.e.}}
\newcommand{\eg}{\emph{e.g.}}
\newcommand{\nan}{{\sf NaN}}
\newcommand{\nil}{{\bf nil}}
\newcommand{\ifif}{{\bf if}}
\newcommand{\ifthen}{{\bf then}}
\newcommand{\ifelse}{{\bf else}}
\newcommand{\ifendif}{{\bf endif}}
\newcommand{\zero}{{\bf 0}}
\newcommand{\one}{{\bf 1}}
\newcommand{\true}{{\sf true}}
\newcommand{\false}{{\sf false}}
\newcommand{\syntax}{{C Syntax}}
%%%%%%

\hyphenation{Suite-Sparse}
\hyphenation{Graph-BLAS}
\hyphenation{Suite-Sparse-Graph-BLAS}

%\input{algobox}

\title{LAGraph: A Community Effort to Collect Graph Algorithms built on top of the GraphBLAS}


\author{
Tim Mattson$^\ddag$, Tim Davis$^\diamond$, Manoj Kumar$^\P$, Ayd\i n Bulu\qc$^\dag$, Scott McMillan$^\S$, Jos\'e Moreira$^\P$, Carl Yang$^{*,\dag}$ \\
\\

 \normalsize
 {\em $^\ddag$Intel Corporation}  {\em $^\dag$Computational Research Division, Lawrence Berkeley National Laboratory} \\
{\em $^\diamond$Texas A and M University} ~~ {\em $\P$IBM Corporation}~~ {\em $\S$Software Engineering Institute, Carnegie Mellon University}   \\
{\em $^*$Electrical and Computer Engineering Department, University of California, Davis}
 }



\begin{document}
\maketitle

\begin{abstract}

In 2013, we released a position paper to launch a community effort to define a 
common set of building blocks for constructing graph algorithms ``in the language of Linear Algebra''.
This led to the GraphBLAS.   We released a specification for the C programming language binding to the
GraphBLAS in 2017.  Since that release, multiple libraries that conform to the 
GraphBLAS C specification have been produced.

In this position paper, we launch the next phase of this ongoing community effort; a project to
assemble a set of high level graph algorithms built on top of the GraphBLAS.  While many 
of these algorithms are well-known with high quality implementations available,
they have not been assembled in one place and integrated with the GraphBLAS.  We call 
this project the LAGraph graph algorithm project and with this position paper, put out a 
call for collaborators to join us.  While the initial goal is to just assemble
these algorithms, we hope in the long run this effort will result production worthy code;
with the LAGraph library serving as an open source repository of verified graph algorithms that use the
GraphBLAS. 

\end{abstract}

\begin{IEEEkeywords}
Graph Algorithms, Linear Algebra, GraphBLAS
\end{IEEEkeywords}

\section{Introduction}
\label{sec:intro}



Graphs are an essential abstraction for a wide range of problems.  There are 
many ways to represent graphs in large graph analytics.  One class of methods defines
graphs in terms of adjacency matrices.
% For example, we can define a graph in terms of an 
% adjacency matrix where the rows and columns are labeled by the vertices and the
% non-zero elements of the matrix are the edges between vertices.  
% 
Expressing graph algorithms as linear algebra expressions ~\cite{kepner2011graph}
is a mature subject.  Multiple 
high performance graph libraries based on sparse 
linear algebra~\cite{combblas,
gadepally2015graphulo, gpi2016, sundaram2015graphmat,che2016programming}
have been developed.   With this ``Graphs as Linear Algebra'' approach established, 
a community of researchers came together to define common building
blocks for graphs expressed in the language of linear algebra.  They launched
this effort with a position paper~\cite{hpec13} in 2013 and formed the GraphBLAS
Forum~\cite{graphblas_web}.  After almost three years of steady work,
the GraphBLAS forum completed the
mathematical formalization of the GraphBLAS~\cite{mathgraphblas16}. With another 
one and a half years of work, the group completed the C
language binding to the GraphBLAS~\cite{cspec}.

Currently there are multiple implementations of the GraphBLAS C specification.  
We have learned a great deal about how to define the operations within the GraphBLAS 
and how to implement them efficiently.  We are now ready to launch the next phase of the project:
to produce a library of high level graph algorithms implemented on top of the GraphBLAS.
We call this library of algorithms \emph{LAGraph}.
Just as we launched the GraphBLAS project with a position paper, we are launching 
this next phase of the project with a position paper.  

The goals of the LAGraph effort include, first and foremost, to bring together the full 
range of known graph algorithms that can be constructed with the GraphBLAS.
From this collection we will be able to systematically assess the coverage of 
graph algorithms based on linear algebra. This will also serve
as raw material in ongoing studies of the fundamental design patterns exploited 
by linear algebra-based graph algorithms. 

The basic outline of the LAGraph project  is summarized in 
Figure~\ref{fig:overview}. The library of algorithms is the single box towards
the middle of the figure.  These algorithms use the GraphBLAS C API
which exists as a separate implementation for a wide range of hardware
targets at the bottom of the figure.  To motivate our work and stay aligned 
with the way data scientists work with graph algorithms, we need to appreciate that
people will use a wide range of languages to access the graph algorithms.  A good library
must be able to work whether called directly from C or indirectly through a wrapper in Python.
Furthermore, the developers of the graph algorithm libraries will need a test harness to
validate the algorithms, I/O utilities, and a build system.  All of these components must
be addressed as part of the LAGraph project. 

LAGraph may not be a production-worthy library in its first incarnation,  
However, we expect the LAGraph effort to produce such a library over time.  
Anticipating that goal, we are constructing
the library ``as if'' it will be used by data analytics end-users. That is, by people
who need the results from graph algorithms with little concern for how they are
implemented.  This requirement of building a library for \emph{end-users} as opposed to 
a library for \emph{graph algorithm researchers} has far reaching implications for the 
design of this software.

\begin{figure}[t]
	\includegraphics[width=\linewidth]{fig/lagraph}
	\caption{\textbf{LAGraph Project Overview.} The project consists of a library of 
	graph algorithms, assorted components to support algorithm development and validation
	(a test harness and I/O utilities) and it must interface to a wide range of languages.
	Underneath these components is a build system, implementations of the GraphBLAS,
	and a variety of hardware targets. \label{fig:overview}}
\end{figure}

We start the paper with a brief summary of the objects and operations defined in
the C specification of the GraphBLAS.  We then  summarize some of the 
early libraries that implement the GraphBLAS C specification.  Next, we describe
the repository where we will build LAGraph.  This is important since the
purpose of a position paper is to attract a community of researchers to join the effort,
which in turn means we want people to understand how to work with and perhaps contribute 
algorithms to the repository.  We then discuss the challenges we faced in writing the 
early version LAGraph and what it suggests about future developments needed in the
GraphBLAS themselves. We close with concluding remarks.
















\section{The GraphBLAS}
\label{sec:math}

%%  <<< Self Plagiarism Warning >>>>
%% The text from here to the end of the operations table was lifted from our
%% GABB paper where we introduced the C API
Consider a graph represented as an $n$-by-$n$ adjacency matrix $\matrix{A}$,
where $A_{ij}$ is the weight of the edge from vertex $i$ to vertex $j$,
and a second $k$-by-$n$ matrix $\matrix{B}$ representing a subset (of size $k$) of the vertices
in the graph, such that $B_{hi}$ is $1$ if the $h$th element of the subset is vertex $i$
(and all other elements of $\matrix{B}$ are 0).  The traditional matrix
product $\matrix{B} \times \matrix{A}$ over real arithmetic of these two matrices returns 
the cost based on the edge weights of reaching the set of vertices
adjacent to the vertices in $\matrix{B}$.  This fundamental operation can be
used to construct a wide range of graph algorithms.

We extend the range of graph operations by keeping the basic
pattern of a matrix-matrix multiplication, but varying
the operators and the interpretation of the values in the matrices (the \emph{domain}).
By carefully choosing operators and the domain, we control the
relation between matrix operations familiar in linear algebra and graph operations, thereby enabling
composable graph algorithms.

In addition to matrix multiplication, the GraphBLAS math specification defines
a range of additional operations over matrices and vectors.  These are summarized in Table~\ref{Tab:GraphBLASOps}.

\begin{table}[h]
\hrule
\begin{center}
\caption{A mathematical overview of the fundamental GraphBLAS operations supported
in the specification. $\matrix{A}$, $\matrix{B}$, and $\matrix{C}$ are GraphBLAS matrices; 
$\vector{u}$, $\vector{v}$, and $\vector{w}$ are GraphBLAS vectors; $i$ and $j$ are single indices;
$\mathbf{i}$ and $\mathbf{j}$ are arrays of indices;
$\oplus$ and $\otimes$ are arbitrary element-wise operators; the element-wise $\odot$
operator is used for the optional accumulation with the output GraphBLAS object where 
$x~\odot\hspace{-0.11cm}= y$ implies $x = x \odot y$; and $F_u()$ is a unary function.
Although not shown here, the input 
matrices $\matrix{A}$ and $\matrix{B}$ may be selected for transposition prior to 
the operation, and masks can be used to control which values are written to the output GraphBLAS object.
\label{Tab:GraphBLASOps}}
\newcommand{\odotequals}{~\odot\hspace{-0.09cm}=\hspace{-0.2cm}}
\begin{tabular}{l|rl}
{\sf Operation name} & \multicolumn{2}{c}{Mathematical description}  \\
\hline
{\sf mxm}          & $\matrix{C}$    $\odotequals$ & $\matrix{A} \oplus.\otimes \matrix{B}$ \\
{\sf mxv}          & $\vector{w}$    $\odotequals$ & $\matrix{A} \oplus.\otimes \vector{v}$ \\
{\sf vxm}          & $\vector{w}^T$  $\odotequals$ & $\vector{v}^T \oplus.\otimes \matrix{A}$  \\
{\sf eWiseMult}    & $\matrix{C}$    $\odotequals$ & $\matrix{A} \otimes \matrix{B}$ \\
                   & $\vector{w}$    $\odotequals$ & $\vector{u} \otimes \vector{v}$ \\
{\sf eWiseAdd}     & $\matrix{C}$    $\odotequals$ & $\matrix{A} \oplus \matrix{B}$ \\
                   & $\vector{w}$    $\odotequals$ & $\vector{u} \oplus \vector{v}$ \\
{\sf reduce} (row) & $\vector{w}$    $\odotequals$ & $\bigoplus_j\matrix{A}(:,j)$  \\
{\sf apply}        & $\matrix{C}$    $\odotequals$ & $F_u(\matrix{A})$ \\
                   & $\vector{w}$    $\odotequals$ & $F_u(\vector{u})$ \\
{\sf transpose}    & $\matrix{C}$    $\odotequals$ & $\matrix{A}^T$ \\
{\sf extract}      & $\matrix{C}$    $\odotequals$ & $\matrix{A}(\vector{i},\vector{j})$ \\
                   & $\vector{w}$    $\odotequals$ & $\vector{u}(\vector{i})$ \\
{\sf assign}       & $\matrix{C}(\vector{i},\vector{j})$  $\odotequals$ &  $\matrix{A}$ \\
                   & $\vector{w}(\vector{i})$  $\odotequals$ & $\vector{u}$ 
\end{tabular}

\end{center}
\hrule
\end{table}

%% << end of self-plagrism >>>

In mapping the GraphBLAS as a set of mathematical operators onto the C programming language
we made a number of fundamental choices~\cite{cspec}.  First, the core data structures required
to represent the objects defined by the GraphBLAS are opaque.   The GraphBLAS API defines a 
contract with the programmer for how these objects will be used, but the implementations and underlying
data structures are left to the implementation.  This opaqueness is critical if the API is to serve diverse hardware
ranging from CPUs to GPUs to specialized graph hardware. Second, we defined a non-blocking  
execution model that allows lazy evaluation.  Ultimately, to optimize sparse linear algebra software we need
to aggressively fuse operations and even restructure algorithms.  This requirement meant that we had to 
carefully define when results from a sequence of GraphBLAS operations must be materialized.

Since the release of the GraphBLAS specification, several implementations of the GraphBLAS have been
developed. These are briefly described below.

% SuiteSparse GraphBLAS:

\subsection{SuiteSparse:GraphBLAS}

SuiteSparse:GraphBLAS is a full implementation of the GraphBLAS standard, which
defines a set of sparse matrix operations on an extended algebra of semirings
using an almost unlimited variety of operators and types.  When applied to
sparse adjacency matrices, these algebraic operations are equivalent to
computations on graphs.  GraphBLAS provides a powerful and expressive framework
for creating graph algorithms based on the elegant mathematics of sparse matrix
operations on a semiring.

The design of a GraphBLAS library is flexible, because its data structures are
opaque to the user.  SuiteSparse:GraphBLAS uses a compressed-sparse vector
storage mechanism, in four different forms.  A matrix can be stored in
row-major order (CSR), or column-major order (CSC).  Each sparse vector
consists of a sorted list of indices, and the corresponding numerical values.
The sparse vectors are packed together into two arrays, and another ``pointer
array'' (of size equal to the dimension of the matrix, say $n$) keeps track of
where each row (or column) starts.  The space taken is $O(n+e)$ for a CSR
matrix with $n$ rows or a CSC matrix with $n$ columns, and with $e$ entries.
Both of these two forms can be modified in a {\em hypersparse} form
\cite{BulucGilbert08}, where the pointer array itself becomes sparse, and
non-empty vectors take no space at all.  The space is reduced to $O(e)$,
however, so that matrices with enormous dimensions can be created, as long as
$e << n$.  Hypersparsity is exploited automatically.  All methods can operate
on all four matrix formats in any combination.

The ability to incrementally modify a graph is critical in many applications.
GraphBLAS includes two operations that can make small incremental changes to a
graph/matrix:  namely \verb'GrB_setElement' and \verb'GrB_assign', and It would
be exceedingly slow to insert or delete a single new entry in a CSR or CSC
format, taking $O(n+e)$ time {\bf per entry} inserted.  Instead, the
non-blocking idea in GraphBLAS is exploited.  Fast deletion of entries is
handled by creating {\em zombies}, which are entries tagged for later deletion.
Fast insertion is handled with {\em pending tuples}, which is a separate
unordered list of $(i,j,a_{ij})$ for each new entry.  When a matrix operation
occurs (such as matrix multiply), all zombies are killed and all pending tuples
are assembled, in a single $O(n+ e + p \log p)$ step (for $p$ pending tuples),
or $O(e +p \log p)$ in the hypersparse case.  As a result, it is just as fast
to use a sequence of $e$ \verb'GrB_setElement' operations to build a matrix, as
it is to create an array of $e$ tuples and use \verb'GrB_build'.  Internally,
SuiteSparse:GraphBLAS is building the list itself, for the user, and then does
a \verb'GrB_build' when the matrix needs to be completed.

To enable high-performance matrix-matrix multiply, a code generation mechanism
is used to build functions for each semiring that can be created with built-in
operators.  The functions can rely on Gustavson's method \cite{Gustavson78}, a
dot product method, and heap-based method, all with masked variants.  A current
prototype of the package adds an early exit mechanism for the MIN, MAX, OR, and
AND monoids, where a dot product can terminate as soon as a terminal value is
found in the result (\verb'true' for OR, for example).  This will enable a fast
direct-optimizing BFS to be written, where the ``pull'' is a dot product, and
the ``push'' a saxpy-based operation (Gustavson's or the heap method).

Since it is meant as the GraphBLAS reference implementation, testing is a vital
component to the package.  In SuiteSparse:GraphBLAS, each GraphBLAS operation
was written twice: once in high-performance algorithms in C, and again in a
very simple and short MATLAB script, using dense matrices with of the required
type.  For example, \verb'GrB_assign' requires about 3,908 lines of C (not
counting comments), but only 161 lines in MATLAB.  The pattern in the MATLAB
version is held as a seperate boolean matrix.  The MATLAB functions are not
intended to be fast.  Instead, they exactly mimic the GraphBLAS API
Specification, line by line, so they can be visually inspected for conformance
to the spec.  For example, matrix multiply is written with a triply-nested
\verb'for' loop.  Then, to test the package, each computation is done in
SuiteSparse:GraphBLAS (via a MATLAB interface) and in the MATLAB mimic.  The
tests pass only if the results are identical in both value and pattern (even
with identical floating-point roundoff error, in most cases).

SuiteSparse:GraphBLAS appears in Debian and Ubuntu linux distros,
and has been released as part of the RedisGraph database
module of the Redis database systems, by RedisLabs, Inc.
\url{https://redislabs.com/redis-enterprise/redis-modules/redis-enterprise-modules/redisgraph/}



% IBM GraphBLAS:
\subsection{IBM GraphBLAS}

%% ToDo: Manoj and Jose

TODO: Manoj and Jose: Describe the IBM GraphBLAS implementation. Citations:\cite{gpi2016,GPI}



% GBTL: GraphBLAS Template Library
\subsection{GBTL: GraphBLAS Template Library}
 
The first version of the GraphBLAS Template Library (GBTL) was written in C++ 
when the GraphBLAS C API project was just beginning.  It was used, in part, 
to study early ideas under discussion in the specification process and was released as a proof of concept 
prior to the finalization of the GraphBLAS API Specification \cite{gbtl-cuda16, McMillan2016}. With the 
release of the GraphBLAS C API Specification~\cite{cspec} in 2017, GBTL was updated to conform to 
the mathematical behavior defined by the specification and released as version 2.0~\cite{gbtl-github}.  
Unlike the C API Specification, GBTL is written in C++ and makes judicious use of templates to 
make the generic aspects of the GraphBLAS specification easier to implement and 
more natural for the C++ programming language. When the GraphBLAS language committee
starts its work on the C++ language binding to the GraphBLAS, GBTL will be submitted as a 
proposed starting point for the discussion.

 
Central to GBTL's design is the concept of a separation of concerns between 
implementation of algorithms written in terms of the GraphBLAS primitives  
and the implementation of those primitives on a targeted hardware platform.   
This separation of concerns is defined by the GraphBLAS API Specification as
illustrated in Figure~\ref{fig:overview}.  
Above this API, GBTL has developed and includes a collection of graph algorithms 
written against its C++ API and has already been shown to be easily translated to 
the C API (compare Figures~\ref{code:pyGB}(c) and~\ref{code:pyGB}(d)).  Below the separation/API, different implementations of the GraphBLAS 
library can be supported for different hardware architectures (referred to as 
``backends'').  In this way we verify that algorithms written once against the API 
can run on different targeted hardware.  One backend that is provided with Version 
2.0 of GBTL implements a mathematically correct version of the C API 
specification and serves as a reference implementation for verifying correctness.  It runs 
in a single thread on a CPU (an earlier version of GBTL also had a GPU 
implementation).  Other backend implementations are currently under development 
to optimize performance, use multiple threads, and to target specialized 
computer architectures.

\subsection{PyGB: python DSL for GraphBLAS}

Another development effort closely related to GBTL is a DSL (domain-specific language) in Python called PyGB~\cite{Chamberlin2016}.  
The goal for PyGB is to closely resemble the GraphBLAS mathematical notation found in the GraphBLAS math spec~\cite{mathgraphblas16}.  
PyGB is a framework is designed and implemented to dispatch dynamically generated and compiled templated 
classes that make calls into native GBTL code.  It demonstrates how Python's syntax and dynamic execution provides 
a high-level abstraction with minimal performance penalty.  While we leave a detailed discussion of pyGB to elsewhere~\cite{Chamberlin2016}
we provide a level-BFS function using pyGB in Figure~\ref{code:pyGB}.

\begin{figure}

\begin{subfigure}{\columnwidth}
  \begin{cplus}
Input: graph, frontier, levels
depth |$\leftarrow$| 0
while nvals(frontier) > 0:
  depth |$\leftarrow$| depth + 1
  levels[frontier] |$\leftarrow$| depth
  frontier<|$\neg$|levels,replace> |$\leftarrow$| graph|$^T$| |$\oplus . \otimes$| frontier
    where |$\oplus . \otimes = \mathbf{\bigoplus} . \mathbf{\bigotimes}($|LogicalSemiring|$)$|
  \end{cplus}
%, |$\oplus  = \mathbf{\bigoplus}($|LogicalSemiring|$)$|
  \caption{Pseudocode}
  \label{subfig:pseudo}
\end{subfigure}\\[1ex]

\vspace{-0.6ex}

\begin{subfigure}{\columnwidth}
  \begin{python}
def bfs(graph, frontier, levels):
  depth = 0
  while frontier.nvals > 0:
    depth += 1
    levels[frontier][:] = depth
    with gb.LogicalSemiring, gb.Replace:
      frontier[~levels] = graph.T @ frontier
  \end{python}
 \caption{PyGB}
\end{subfigure}\\[1ex]

\begin{subfigure}{\columnwidth}
  \begin{cplus}
template<class Mat, class Frontier, class Levels>
void bfs(Mat &graph, Frontier frontier, Levels &levels)
{
  GrB::IndexType depth = 0;
  while (frontier.nvals() > 0) {
    ++depth;
    GrB::assign(levels, frontier, GrB::NoAccumulate(),
                depth, GrB::AllIndices(), false);
    GrB::mxv(frontier, GrB::complement(levels),
             GrB::NoAccumulate(),
             GrB::LogicalSemiring<GrB::IndexType>(),
             GrB::transpose(graph), frontier, true);
  }
}
\end{cplus}
\caption{GBTL \cpp}
\end{subfigure}

\begin{subfigure}{\columnwidth}
  \begin{cplus}
void bfs(GrB_Matrix  graph,
         GrB_Vector  frontier,
         GrB_Vector *levels)
{
  GrB_Index n, nvals;
  GrB_Matrix_nrows(&n, graph);
  GrB_Vector_nvals(&nvals, frontier);
  GrB_Semiring LogicalSemiring;
  GrB_Descriptor Desc_TranA_ScmpM_Replace;
  //...
  GrB_Index depth = 0;
  while (nvals > 0) {
    ++depth;
    GrB_assign(*levels, frontier, GrB_NULL,
               depth, GrB_ALL, n, GrB_NULL);
    GrB_mxv(frontier, *levels, GrB_NULL,
            LogicalSemiring, graph, frontier
            Desc_TranA_ScmpM_Replace);
    GrB_Vector_nvals(&nvals, frontier);
  }
}
\end{cplus}
\caption{GraphBLAS C API}
\end{subfigure}

\caption{Level-based BFS traversal in math pseudocode, PyGB, GBTL \cpp, and using the GraphBLAS C API.\label{code:pyGB}}
\end{figure}


Notice how the meaning of the code
is straightforward since the DSL closely tracks the notation from the GraphBLAS math spec.  We believe in the long run, the 
future of graph algorithms will depend heavily on such DSLs.

%\begin{CodeExample}
%{\textbf{Level-BFS function in pyGB}}
%{code:pyGB}
%\begin{lstlisting}
%def bfs(graph, frontier, levels):
%    level = 0
%    while frontier.nvals > 0:
%        level += 1
%        levels[front][:] = level
%        with gb.LogicalSemiring, gb.Replace:
%            frontier[~levels] = graph.T @ frontier 
%\end{lstlisting}
%\end{CodeExample}



% Gunrock GraphBLAS:
\subsection{Gunrock GraphBLAS}

%% ToDo: Carl

TODO: Carl: Describe the GunRock implementation





\section{LAGraph Repository}
\label{sec:repo}

The hypothesis underlying the GraphBLAS is that algorithm designers can focus on expressing their algorithms
in terms of the high level linear algebra operations defined in the GraphBLAS while leaving low level
optimizations to any particular hardware platform to the implementation of the GraphBLAS.  Ultimately, 
we want hardware vendors to be responsible for creating highly tuned versions of the GraphBLAS
specialized to the features of their systems.
Kumar et al.~\cite{KuHPEC2016} show that mitigating the adverse impact of memory
latency on performance of algorithms for large graph can lead to significant improvements.
Such optimizations require detailed knowledge of the underlying system, and hence the 
the low level optimizations are best left to hardware vendors.
Furthermore, a tuned linear algebra library delivers
better performance than straight-forward textbook implementation on many basic graph algorithms~\cite{KuJour}.

Algorithm designers will naturally wonder how much performance is lost due to the use of a high 
level API such as the GraphBLAS.  As shown in~\cite{KuJour}, a linear algebra 
implementation brings inherent efficiency advantages to 
graph algorithms due to the more structured access to data afforded by the linear algebra 
formulation~\cite{KuJour}.
The GraphBLAS API is more general, but we expect its implementations to retain or improve upon
the efficiency advantages.
This is an untested hypothesis, since until now, we have not had multiple 
implementations of the GraphBLAS API tuned to the features of a range of platforms.

Testing this hypothesis of the performance potential afforded by the GraphBLAS is a major outcome
we anticipate from the LAGraph project.  By collecting high level graph algorithms and validating them
across an engaged community, we will produce the library of algorithms needed to evaluate the 
effectiveness of the GraphBLAS approach.

Before we can conduct such experiments, however, we need to collect graph algorithms implemented on 
top of the GraphBLAS. We have created a git hub repository at
\url{https://github.com/GraphBLAS/LAGraph} for members of the LAGraph community to 
use to contribute GraphBLAS algorithms. The basic elements of the repository include:
%% DONE: this is complete:
\begin{itemize}
\item A Build System for creating the LAGraph library and the test routines.
\item A library of utilities including loading matrices from disk in Matrix Market format~\cite{MM},
    evaluating results, and creating random test matrices.
\item A directory of graph algorithms.
\item A directory holding a test harness for each algorithm.
\end{itemize}

We will write documentation, a programmer's reference guide and define procedures for how people
can add new algorithms.  


\section{Discussion}
\label{sec:disc}

We are early in the LAGraph project.  At this point, we've defined the basic structure of the
repository and the overall goals of the project. We have built an early framework for testing
and core utility routines to support software development.  Finally, we assembled a few algorithms which we
are using to test the basic structure of the software system.

Even at this early phase of the project, we have learned a great deal about how the GraphBLAS
will interact with end-users.  The objects manipulated by the GraphBLAS are opaque.  An implementation
needs complete freedom in how data structures underlying the GraphBLAS are implemented.  
A graph algorithm, however, is used as part of a processing pipeline.  For example, data may
exist in data frames.  A subset of the data is collected and filtered to produce relationships
represented by a graph. Properties of the graph are computed and based on the result 
a new branch in the processing pipeline may be accessed.  

The key here is that graphs inside the GraphBLAS are opaque, but externally they are 
anything but opaque.  This suggests that we need to define functions to import and export 
data in standard sparse array formats into LAGraph.   The initial thinking was that this
import functionality would be part of LAGraph and not the GraphBLAS.  The only way to 
do that, however, is if we repacked the input sparse format into separate arrays for column 
indices, row indices, and values and then use GrB\_build\_matrix() to construct the graphBLAS
matrix object.  This would extremely inefficient.  We need a way to directly import arrays 
in standard sparse formats into the GraphBLAS and since the GraphBLAS data types are
opaque, this can only be done as a GraphBLAS routine.  

Graph algorithms do not occur in isolation.   The LAGraph library, therefore, needs to 
return a handle to an opaque GraphBLAS object so it can be used without incurring copy overhead
in subsequent graph operations.  Given the nonblocking execution model, this raises interesting
design questions about how memory consistency between the library and the application is
managed.  

Graphs can be quite large.   Hence, we do not believe the default mode should be that 
sparse arrays input to LAGraph are copied into a separate memory region to hold the 
opaque GraphBLAS object.  We believe it is important that the memory for the input array
be used to hold the graphBLAS object. This means the input array is destroyed and replaced
with the graphBLAS opaque object.  Since during a computation it is common that elements of a
matrix fill-in, the GraphBLAS routines will in some cases need to allocate additional memory during a 
calculation.  This may require deallocating memory and reallocating a larger block of memory.
This means we will likely encounter situations where a region of memory allocated in an application
is deallocated inside a library routine.  This violates the separation of concerns between application
and library code expected in well engineered software. There is also the question of communicating
to the library routine how the input sparse array was allocated in the first place so the right deallocator 
can be used.

The point of these issues is that in designing an effective library, there are a host of 
complicated issues to resolve.  A major part of the research contribution of this project will
be how we solve this complicated problems.




\section{Conclusion}
\label{sec:conclusion}

The GraphBLAS forum started its work in 2013 to standardize the building blocks for 
graph algorithms formulated from linear algebra expressions~\cite{hpec13}.  We now
have a C specification for the GraphBLAS~\cite{cspec} and multiple 
implementations~\cite{Davis19,GPI,gbtl-github}.   The next 
step in this journey is to define a library of high level graph algorithms
that are based on the GraphBLAS.  

The GraphBLAS was a community effort launched by a position paper.
This is a position paper to launch LAGraph project, our community effort to 
collect and validate 1) a set of high quality basic graph algorithms that run 
on top of the GraphBLAS, and 2) support libraries for development of graph analytics 
applications.  Example of support libraries are I/O, generation of scale-free graphs,
basic measurements on graphs, and changing representation of graphs.  
We urge readers interested in joining us as we
work on LAGraph to contact any of the authors of this position paper.




\section*{Acknowledgments}

Tim Davis would like to acknowledge the support of SuiteSparse:GraphBLAS by MIT
Lincoln Laboratory, Intel, and the National Science Foundation (NSF
CNS-1514406).

TODO: Ayd\i n Bulu\qc\, Carl Yang and maybe Scott will need to acknowledge someone.

\bibliographystyle{IEEEtran}
\bibliography{LAGraph19}

\end{document}  
