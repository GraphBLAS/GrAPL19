\subsection{IBM GraphBLAS}

\newcommand{\code}[1]{{\tt #1}}

The IBM GraphBLAS implementation was announced at IPDPS 2018 and made
available at \code{https://github.com/IBM/ibmgraphblas}, fulfilling
the GraphBLAS C API requirement of two conforming implementations,
and promoting that specification from \emph{provisional} to
\emph{definitive}. The approach adopted by the development team was
heavily influenced by their experience with IBM's Graph Programming
Interface~\cite{gpi2016,GPI}.

Among the various objectives of the IBM implementation, we note the
desire to experiment with an implementation that allows multiple data
representations and a layered approach to algorithms, keeping a GraphBLAS
API layer on top of a more fundamental \emph{back-end} layer that performs
the heavy computation.

Use and operation of the IBM GraphBLAS is straightforward.
The application programmer has access to a C11-compliant include
file (\code{GraphBLAS.h}) that defines the API according to the
specification. This include file exposes nothing of the internals of
the run-time. The run-time itself is written in C++ and packaged as a
library (\code{libibmgraphblas.so}) with C language bindings.  The choice
of C++ as the implementation language has led to a simple and concise
specification-compliant implementation.

One of the jobs of the \code{GraphBLAS.h} include file is to
convert the polymorphic version of the API into the nonpolymorphic
one. The nonpolymorphic methods are then directly supported by the
library. Conversion of the polymorphic interface is accomplished through
standard C preprocessor features, primarily in supporting number of
arguments polymorphism, in combination with the standard C11 language
\code{\_Generic} construct to support type polymorphism.

As previously mentioned, the IBM GraphBLAS library is implemented in
C++. The API methods are declared to have a C interface, so that C user
programs can bind to them as specified. Objects internal to the library
are declared as C++ classes, with member methods doing the actual work.
We want to emphasize that this is a C++ implementation of a C API, and
not an API for GraphBLAS that exploits C++ features, as other efforts
are pursuing~\cite{gbtl-github}.

The contents of a GraphBLAS vector object are implemented using standard
C++ containers.  An unordered set of indices (\code{uset<GrB\_Index>})
is used to represent the structure of the vector, while an unordered
map of indices to pointers (\code{umap<GrB\_Index,void*>}) represents
the elements.  (Currently, \code{uset} and \code{umap} are just renames
for the \code{std::unordered\_set} and \code{std::unordered\_map} standard
containers of C++, but one could use more specialized implementations.)

Similarly, the contents of a matrix are represented both as a standard
C++ container vector of rows and a standard C++ container vector of
columns. Accessor methods enforce consistency of both representations.

In the current IBM GraphBLAS, all methods are fully blocking. That is,
the methods return only after all computations are fully performed (or
an error is detected).  Since the GraphBLAS nonblocking mode allows for
deferred execution but does not require it, this behavior complies with
the specification.

We finish this brief overview of the IBM GraphBLAS with a discussion
of error handling, as we believe it reflects an important aspect of the
interoperability between a C API and a C++ implementation.  The GraphBLAS
C API defines two kinds of errors: API errors and execution errors.
API errors reflect incorrect usage of the API (for example, passing
parameters that are not valid or consistent). Execution errors indicate
that something went wrong during the actual execution of a method,
and can be caused either by programmer error or by environment issues
(for example, not enough memory).

In the IBM GraphBLAS run-time, API errors are detected through
explicit checks in the implementation of each method in the API layer
(the \emph{front-end}).  Execution errors, on the other hand, occur
inside the member methods of the various objects internal to the library
(the \emph{back-end}).  Those methods, following standard C++ practice,
use exceptions to signal an error. To transform those exceptions into
proper GraphBLAS C API return codes, the body of each GraphBLAS API
method is wrapped by a \code{try}/\code{catch} block, which then returns
the GraphBLAS execution error code corresponding to the caught exception.
