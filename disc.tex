\section{Discussion}
\label{sec:disc}

We are early in the LAGraph project.  At this point, we've defined the basic structure of the
repository and the overall goals of the project. We have built an early framework for testing
and core utility routines to support software development.  Finally, we assembled a few algorithms which we
are using to test the basic structure of the software system.

Even at this early phase of the project, we have learned a great deal about how the GraphBLAS
will interact with end-users.  The objects manipulated by the GraphBLAS are opaque.  An implementation
needs complete freedom in how data structures underlying the GraphBLAS are implemented.  
A graph algorithm, however, is used as part of a processing pipeline.  For example, data may
exist in data frames.  A subset of the data is collected and filtered to produce relationships
represented by a graph. Properties of the graph are computed and based on the result 
a new branch in the processing pipeline may be accessed.  

The key here is that graphs inside the GraphBLAS are opaque, but externally they are 
anything but opaque.  This suggests that we need to define functions to import and export 
data in standard sparse array formats into LAGraph.   The initial thinking was that this
import functionality would be part of LAGraph and not the GraphBLAS.  The only way to 
do that, however, is if we repacked the input sparse format into separate arrays for column 
indices, row indices, and values and then use GrB\_build\_matrix() to construct the graphBLAS
matrix object.  This would extremely inefficient.  We need a way to directly import arrays 
in standard sparse formats into the GraphBLAS and since the GraphBLAS data types are
opaque, this can only be done as a GraphBLAS routine.  

Graph algorithms do not occur in isolation.   The LAGraph library, therefore, needs to 
return a handle to an opaque GraphBLAS object so it can be used without incurring copy overhead
in subsequent graph operations.  Given the nonblocking execution model, this raises interesting
design questions about how memory consistency between the library and the application is
managed.  

Graphs can be quite large.   Hence, we do not believe the default mode should be that 
sparse arrays input to LAGraph are copied into a separate memory region to hold the 
opaque GraphBLAS object.  We believe it is important that the memory for the input array
be used to hold the graphBLAS object. This means the input array is destroyed and replaced
with the graphBLAS opaque object.  Since during a computation it is common that elements of a
matrix fill-in, the GraphBLAS routines will in some cases need to allocate additional memory during a 
calculation.  This may require deallocating memory and reallocating a larger block of memory.
This means we will likely encounter situations where a region of memory allocated in an application
is deallocated inside a library routine.  This violates the separation of concerns between application
and library code expected in well engineered software. There is also the question of communicating
to the library routine how the input sparse array was allocated in the first place so the right deallocator 
can be used.

The point of these issues is that in designing an effective library, there are a host of 
complicated issues to resolve.  A major part of the research contribution of this project will
be how we solve this complicated problems.


