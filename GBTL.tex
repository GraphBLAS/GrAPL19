\subsection{GBTL: GraphBLAS Template Library}
 
The first version of the GraphBLAS Template Library (GBTL) was written in C++ 
when the GraphBLAS C API project was just beginning.  It was used, in part, 
to study early ideas under discussion in the specification process and released as a proof of concept 
prior to the finalization of the GraphBLAS API Specification \cite{gbtl-cuda16, McMillan2016}. With the 
release of the GraphBLAS C API Specification~\cite{cspec} in 2017, GBTL was updated to conform to 
the mathematical behavior defined by the specification and released as version 2.0~\cite{gbtl-github}.  
Unlike the C API Specification, GBTL is written in C++ and makes judicious use of templates to 
make the generic aspects of the GraphBLAS specification easier to implement and 
more natural for the C++ programming language. When the GraphBLAS language committee
starts its work on the C++ language binding to the GraphBLAS, GBTL will be submitted as a 
proposed starting point for the discussion.

 
As illustrated in Figure~\ref{fig:gbtl}, central to GBTL?s design is the concept of a separation 
of concerns between implementation of algorithms written in terms of the GraphBLAS primitives 
and the implementation of those primitives on a targeted hardware platform.   Today the
 separation of concerns is defined by the GraphBLAS API Specification.  Above 
 this API, GBTL has developed and includes a collection of graph algorithms 
 written against its C++ API and has already been shown to be easily translated to the C API.  
 Below the separation/API, different implementations of the GraphBLAS library can be supported 
 for different hardware architectures (referred to as ?backends?).  In this way we verify that algorithms 
 written once against the API can run on different targeted hardware.  In terms of backend 
 implementations, a mathematically correct reference implementation (not optimized) for single 
 threaded CPU is provided with Version 2.0 of GBTL (version 1.0 also had a GPU implementation).  
 Other backend implementations are currently under development to target specialized computer architectures.

\begin{figure}[t]
\includegraphics[width=\linewidth]{gbtl}
\label{fig:gbtl}
\caption
{\textbf{Graph processing software stack.}}
\end{figure}
%
%  I don't know how to make the listing work below so I will comment this out for now
%Another development effort closely related to GBTL is a DSL in python called pyGB~\cite{Chamberlin2016}.  
%The goal for pyGB is to closely resembles the GraphBLAS mathematical notation found in the GraphBLAS math spec~\cite{mathgraphblas16}.  
%PyGB is framework is designed and implemented to dispatch dynamically generated and compiled templated 
%classes that make calls into native GBTL code.  It demonstrates how Python's syntax and dynamic execution provides 
%a high-level abstraction with minimal performance penalty.  While we leave a detailed discussion of pyGB to elsewhere~\cite{Chamberlin2016}
%we provide a depth-BFS function using pyGB in figure~\ref{code:pyGB}.  Notice how the meaning of the code
%is straightforward since the DSL closely tracks the notation from the GraphBLAS math spec.  We believe in the long run, the 
%future of graph algorithms will depend heavily on such DSLs.
%
%\begin{CodeExample}
%{\textbf{Depth-BFS function in pyGB}}
%{code:pyGB}
%\begin{lstlisting}
%def bfs(graph, frontier, levels):
%    depth = 0
%    while frontier.nvals > 0:
%        depth += 1
%        levels[front][:] = depth
%        with gb.LogicalSemiring, gb.Replace:
%            frontier[~levels] = graph.T @ frontier 
%\end{lstlisting}
%\end{CodeExample}


